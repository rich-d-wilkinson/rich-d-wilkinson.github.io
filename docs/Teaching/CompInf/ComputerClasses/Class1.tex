
\documentclass[12pt,a4paper]{article}

\setlength{\topmargin}{-2cm} \setlength{\topskip}{0cm}
\setlength{\headheight}{1.5cm} \setlength{\headsep}{1cm}
\setlength{\textheight}{24.5cm} \setlength{\oddsidemargin}{.25cm}
\setlength{\evensidemargin}{.25cm} \setlength{\textwidth}{17cm}
%\setcounter{tocdepth}{1}
\setlength{\unitlength}{1cm}
%\input /home3/st/st1jeo/tex/epsf.tex

\input{../epsf.tex}
\newcommand{\balpha}{\mbox{\boldmath{$\alpha$}}\unboldmath}
\newcommand{\bn}{\mbox{$\mathbf{n}$}}
\newcommand{\ba}{\mbox{$\mathbf{a}$}}
\newcommand{\bx}{\mbox{$\mathbf{x}$}}
\newcommand{\xvec}{\mbox{$\bx_1,\ldots,\bx_n$}}
\newcommand{\yxvec}{\mbox{$y_1=\eta(\bx_1),\ldots,y_n=\eta(\bx_n)$}}
\newcommand{\by}{\mbox{$\mathbf{y}$}}
\newcommand{\boldm}{\mbox{$\mathbf{m}$}}
\newcommand{\bh}{\mbox{$\mathbf{h}$}}
\newcommand{\bt}{\mbox{$\mathbf{t}$}}
\newcommand{\bbeta}{\mbox{\boldmath{$\beta$}}\unboldmath}
\newcommand{\bdelta}{\mbox{\boldmath{$\delta$}}\unboldmath}
\newcommand{\btheta}{\mbox{\boldmath{$\theta$}}\unboldmath}
\newcommand{\bo}[1]{\mbox{$\mathbf{#1}$}}
\newcommand{\yxpvec}{\mbox{$y_1'=\eta(\bx_1'),\ldots,y_{n'}'=\eta(\bx_{n'}')$}}
\newcommand{\xpvec}{\mbox{$\bx_1',\ldots,\bx_{n'}'$}}
\newcommand{\ri}{\mbox{$^{131}I$} }
\newcommand{\bX}{\bo{X}}
\newcommand{\bY}{\bo{Y}}
\newcommand{\betahat}{\mbox{$\hat{\bbeta}$}}
\newcommand{\sighat}{\mbox{$\hat{\sigma} ^2$}}
\newcommand{\gap}{ \\ \\ \\ \\ \\ }
\newcommand{\vect}[1]{\mbox{{\boldmath $#1$}}}
\renewcommand{\baselinestretch}{1.25}
% Standardized forms for name of S-plus
\renewcommand{\S}{R}
\newcommand{\Shead}{R}

\pagestyle{myheadings} \markboth{Programming exercises 1}{Programming exercises 1}

\usepackage{amsmath}
\usepackage{amsfonts}
\usepackage{amssymb}
\usepackage{epstopdf}
\usepackage{graphicx}
\newtheorem{definition}{Definition}

\title{Exercises: Computer Lab 1}
\graphicspath{ {../Graphs/} }
\begin{document}
%\section*{Programming exercises}
\maketitle

%ADD THE EXERCISES FROM CHAPTER 1.


I highly recommend that you use Rstudio as your R GUI. I also recommend that you learn to use R Markdown as a way to document your code and to write your coursework solutions. You can get started with R Markdown by clicking File $\rightarrow$ New File $\rightarrow$ R Markdown and following the instructions.

%First get the R workspace \verb"MAS472.RData" from MOLE, and load it
%into R. This contains data for questions 3 and 6, and example
%solutions for question 1.
\begin{enumerate}
\item 

Suppose $X_1, \ldots, X_{10}$ are iid $U[0,1]$ random variables. Find the distribution of 
$$R(X) = \max_i\{X_i\} - \min_i \{X_i\}.$$
What is $\mathbb{P}(R(X) > 0.99)$?

If you are struggling, note that the code from the problems discussed in lectures is available on MOLE.


\item Estimate the integral 
$$I = \int_{0}^{10}\frac{1}{(1+x^2)} dx$$ 
using at least two different choices for the proposal density  $g(\cdot)$.

Find 95\% confidence intervals for your estimates using the central limit theorem.

%Note that you can check your confidence inis appropriate by generating 1000 different estimates of I, and checking that the o

\item Let 
$$I_1 = \int_0^1 e^{-x^2} {\rm d} x \quad \mbox{and}\quad I_2 = \int_0^1 (\cos(50x) + \sin(20x))^2 {\rm d} x$$

Estimate both of these integrals using Monte Carlo (using an estimator of your choice). Show that the root mean square error of your estimator scales as $O(n^{-1/2})$. To do this, you will need to repeat the analysis multiple times for a range of values of $n$ (i.e., for $n=10,50,100,500,1000,5000,10000$ estimate the integral 100 times and calculate the standard error).

Calculate these integrals using the mid-ordinate rule. Show that the error now scales as $O(n^{-2})$. Note that because the mid-ordinate rule is very accurate for 1d intervals, you should only consider values of $n$ between (2 and 100 say).


\end{enumerate}

\end{document}